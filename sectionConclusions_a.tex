\section{Conclusions and Future Work}

The proof of concept implementation of corpus-based convolution efficiently combines the intuitive and expressive control of dynamics and temporal shape provided by hand-drumming on a surface, with the rich and varied timbres chosen from large corpora of sound by content-based selection.

% allows to articulate grains chosen from a

%One possible drawback of the prototypical implementation described here is that the 
One important part of the digital musical interface has not been treated specifically: How to control the selection.  In its simplest form, the selection is given by a 2D target position via a positional controller such as the mouse, XY-pad (graphics tablet or touch screen) (cf. \cite{Schwarz-nime2012-sound-space}, positional control).
This means however that one hand will be used for the position input and only one hand is left for contact interaction.

We do have ideas for future developments how to control navigation by other means in order to have both hands free to play the contact microphones:

\begin{description}
\item[Hand position tracking] either by 2D or 3D camera or audio triangulation would perfectly integrate selection of grains by the position of the hands and their expressive play.  Here, the interaction space should be optimised by evenly distributing grains over the whole playing surface \cite{LallemandSchwarz-dafx2011-distribute}.
\item[Audio mapping], similar to what was used in the \verb|bonk~| trigger-based previous system, would extrapolate the dynamics and timbre of the contact interaction sound to not only articulate a grain but also change the selection.
\item[Voice control] would use the performer's voice timbre and morphology to select grains by similarity matching combined with an adaptive projection that maps the space of possible voice timbres to a given corpus' descriptor space \cite{StowellPlumbley-smc2010-timbre-remapping-regression-tree,Fasciani-si2013b}.
\end{description}

\textbf{proof of concept, many things to be added and optimised}

% - amplify input signals, not grain outputs
Possible optimisations are, first, to unify the gain changes due to mixing and due to the handling of the release by incorporating both into the amplification of the input signal before the convolution.
% - optimisation: mix ffts, not grains
Second, the partitioned convolution algorithm used in HISSTools multiplies FFT spectra of the input signal and the impulse response to carry out the convolution more efficiently.  If we now constitute the impulse response by interpolating between the 3 closest grains by mixing their FFTs, we only need one convolution instead of 3 separate ones.  The grain FFTs could even be precomputed for the whole corpus. % and zero-padded to the longest grain size.


\textbf{(We could also explore the bass being convolved in line with our previous paper it could help the problem of attack / latency /granularity mentionned in the paper \cite{TremblaySchwarz-nime2010-surfing-the-waves}, Diemo)}
