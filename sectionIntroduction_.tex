\section{Introduction}


Contact gestures are an intuitive way to express musical rhythm and dynamics on any surface or object by hitting it, scratching it, strumming it, etc.  It is striking to observe both how easily one can express a dynamic rhythm by table-drumming, one's hands providing a wide range of timbres through the use of fingernails, fingertips, knuckles, and thumb ball, and also how subtle the nuances of timbre and dynamic can be from the onset of the exploration.

The most obvious limiting factor in both sound and variety remains the surface that is hit, and this is where digital musical instruments (DMIs) can contribute to the enrichment of the sonic outcome while keeping the nuances, expressivity, and fully embodied interaction of hand contact gestures, thus turning any surface into an expressive musical instrument.

In this paper, we propose a new way to combine the sonic richness of large corpora of sound (easily accessible via navigation through a space of sound descriptors), with the intuitive gestural interaction on a surface, by capturing the contact interaction sounds with piezo or contact microphones. The captured sound is then convolved with an impulse response (IR) that is, in fact, a sound grain taken from a corpus of sounds such that the grain's timbre and micro-structure is imprinted onto the microphone's signal.
Seen the other way around, the grain's timbre is articulated by the spectro-morphology of the contact interaction sound captured by the microphone. 

The choice of grain is made by corpus-based concatenative synthesis' content-based approach to navigation of large databases of sound, as implemented, for instance, by the CataRT system\footnote{\url{http://imtr.ircam.fr/index.php/CataRT}}, where all snippets of sound are laid out in a 2D space according to their sonic characteristics.  This 2D space can be navigated with an appropriate 2D controller (mouse, XY-pad, accelerometer, joystick, motion capture) to choose and to smoothly mix grains to be excited by the microphone input.

This paper presents the early conclusions of what is a simple way to join two existing proven technologies in order to multiply respectively their timbral richness and gestural expressivity. It will be evaluated in section \ref{sec:eval} as a proof of concept with further ideas for potential development in section~\ref{sec:future}.

% The quest for expressive play of digital musical instruments (DMIs) ofter leads to the use of contact microphones

% Contact microphones are an efficient means for expressive... \textbf{(mit Literaturverzeichnis bitte ;-)}


%%% Local Variables:  
%%% mode: latex 
%%% TeX-master: "../SchwarzHarkerTremblay-nime2014-corpus-convolution" 
%%% End: 
