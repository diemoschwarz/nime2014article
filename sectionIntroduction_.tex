\section{Introduction}


Contact gestures are an intuitive way to express musical rhythm and dynamics on any surface or object by hitting, scratching, strumming.  It is striking to observe how easily one can express a dynamic rhythm by table-drumming, the hands giving a range of timbres using the fingernails, fingertips, knuckles, thumb ball.

The limiting factor in both sound and variety remains the surface that is hit, and it is here that digital musical instruments (DMIs) can contribute to enrich the sonic outcome while keeping the nuances, expressivity, and fully embodied interaction of hand contact gestures, thus turning any surface into an expressive musical instrument.

We propose a new way to combine the sonic richness of large corpora of sound, easily accessible via navigation through a space of sound descriptors, with the intuitive gestural interaction on a surface by capturing the contact interaction sounds with piezo or contact microphones.

The sound is then convolved with an impulse response (IR) taken from a sound grain within a corpus of sounds such that the grain's timbre and micro-structure is imprinted onto the microphone output, or seen the other way around, the grain's timbre is articulated in dynamics and time by the contact interaction sound captured by the microphone. 

The choice of grain is taking advantage of corpus-based concatenative synthesis' content-based approach to navigation in large databases of sound as embodied, for instance, by the CataRT system\footnote{\url{http://imtr.ircam.fr/index.php/CataRT}}, where all snippets of sound are laid out in a 2D space according to their sonic characteristics.  This 2D space can be navigated with an appropriate 2D controller (XY-pad, accelerometer, joystick, motion capture) to choose and to smoothly mix grains.

This paper presents the early conclusions of what is a simple way to join two existing proven technologies in order to multiply respectively their timbral and gestural expressivity. It will be evaluated in section \ref{sec:eval} as a proof of concept with further ideas for potential development in section~\ref{sec:future}.

% The quest for expressive play of digital musical instruments (DMIs) ofter leads to the use of contact microphones

% Contact microphones are an efficient means for expressive... \textbf{(mit Literaturverzeichnis bitte ;-)}
