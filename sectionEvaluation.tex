\section{Evaluation}\label{sec:eval}

The first thing that strikes the users of the system is the immediacy of the instrument: the excitation of the grains with the piezo's sound complex gestures is direct, rich, palpable, without any noticable latency. A complex dynamic between both hands, namely the percusive one and the corpus-navigator one, quickly establish itself and allows many games: with more percussive grains, the piezo becomes a filter, adding texture; with more percussive piezo gestures, the grains pop to life in a very rich way, while retaining their temporal micro-structure.


The absence of latency makes this method immediately superior to the \verb|bonk~| method. Moreover, the infinite nuances of the piezo input are cast upon the grains, giving it infinitely more depth.

Compared to the bass guitar musaicing \cite{TremblaySchwarz-nime2010-surfing-the-waves}, what stands out again is the new method's lack of latency, as well as its ability to keep an infinite range of nuances by the complex hybridisation of the signals. Moreover, it confirms the two issues presented in the previous paper, namely the granularity of the sound and the attack problems: in the new method, it seems that the complexity of the signal combination mitigates the granular effect, giving it a much smoother articulation.

As this is a proof of concept, the authors are eager to share the results in a usable manner, allowing the DMI designers and users to experience the results for themselves. The code is therefore available at (link removed for anonymity) with all its dependencies. For the reader who would not want to run the code, sound examples of this new corpus-based convolution method are also available on \url{http://demos.concatenative.net/}.
