\section{Early Evaluation}\label{sec:eval}

As this paper is a proof of concept, the authors are eager to share the results in a usable manner, allowing the DMI designers and users to experience the results for themselves. The code is therefore available at PERMALINK with all its dependencies. What follows is a short yet promising early evaluation.

When the new system is played in comparison with the previous \verb|bonk~|-based instance, the first stricking improvement is the immediacy of the instrument: the excitation of the grains with the piezo's sound complex gestures is direct, rich, palpable, without any noticable latency. A complex dynamic between both hands, namely the percusive one and the corpus-navigator one, quickly establish itself and allows many games: with more percussive grains, the piezo becomes a filter, adding texture; with more percussive piezo gestures, the grains pop to life in a very rich way, while retaining their temporal micro-structure.

In order to illustrate the difference, the reader is invited to download the instrument. For the reader who would not want to run the code, a short comparison film has been made available at PERMALINK, as well as further sound examples of this new instrument. Note that the left channel is the unprocessed piezo signal, and the right channel is the resulting output of the DMI. In the first video, we can hear the same gesture performed by itself, then on the old \verb|bonk~|-based DMI, and finally on the new DMI, without then with pre-emphasis. As the examples show, the rich palette of nuances of the piezo input are cast upon the grains, giving it more depth and spectral contour. Moreover, but sadly difficult to assess from the video, the absence of latency makes the new DMI more immediate than the previous one.

Compared to the bass guitar musaicing described above \cite{TremblaySchwarz-nime2010-surfing-the-waves}, what stands out again is the new method's absence of latency, as well as its ability to keep a wide range of nuances, as produced by the complex hybridisation of the signals. Moreover, it confirms and solves the two issues of the musaicing instrument, as presented in its paper, namely the granularity of the sound and the attack problems: in the new method, it seems that the complexity of the signal combination mitigates the granular effect, giving it a much smoother articulation.
