\section{Related Work}


\subsection{Interaction by Contact Gestures}


Expressive play of digital musical instruments (DMIs) via contact microphones on arbitrary surfaces has entered the spotlight through the \name{Mogees} project\footurl{http://www.brunozamborlin.com/mogees/}  \cite{Zamborlin14a}.  It excites physical models of string or bell resonators with the input of the microphone, allowing to hit, scratch, and strum any surface and turn it into a musical instrument.  


That work was based on research in the real-time music interaction team at Ircam \cite{Rasamimanana11a,Bevilacqua11b,Zamborlin14a}, that lead to the \name{MO} modular musical objects, winner of the 2011 Guthmann competition \url{xxx guthman video}.


The first author has been using piezo pickups since 2009 on various surfaces, that allow to hit, scratch, and strum the corpus of sound \cite{Schwarz-nime2012-sound-space}, exploiting
its nuances according to the sound of the impacts which is analysed and mapped to
the 2D navigation space of \catart\.


The approach here uses an attack detector (\code{bonk$\sim$}) that also outputs the spectrum of the
attack audio frame in 11 frequency bands.  Total energy and centroid of this spectrum is mapped to
the~x and~y target position in the 2D interface to select the units to play from the corpus.
This means, for instance, dull, soft hitting plays in the lower-left corner, while sharp, hard hitting plays more in the upper right corner.


The drawbacks of this method is that the attack detection is not 100\% accurate and introduces some latency, but since the signal from the piezos is mixed to the
audio played by \catart, the musical interaction still works.


This approach is documented on \url{http://imtr.ircam.fr/imtr/CataRT_Instrument} in videos~8.1 and~8.2.


\subsection{\CBCS}


number of sound descriptors, which describe their sonic characteristics.
These descriptors are typically pitch, loudness, brilliance, noisiness, roughness, spectral shape, or meta-data, like instrument class, phoneme label, that are attributed to the units,
and also include the segmentation information of the units.
These sound units are then stored in a database (the \name{corpus}).  For synthesis, units are
selected from the database that are closest to given \name{target} values for some of the
descriptors, usually in the sense of a weighted Euclidean distance.
The selected units are then concatenated (overlapped) and played, after possibly some transformations.


How a musician can interact with and play the corpus as a musical instrument has been the topic of a recent article \cite{Schwarz-nime2012-sound-space} that shows that the central notion of the interaction is the sound space as an interface to CBCS through which the musician navigates with the help of gestural controllers.  
The actual instrument is determined by the controller that steers the
navigation, which fall into the groups of positional control, and control by the analysis of audio
input.


\subsection{(Partitioned) Convolution}


the HIRT, the HISS, and the hoobledyhoo (especially the latter)
